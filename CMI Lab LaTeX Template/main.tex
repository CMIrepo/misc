\documentclass{CMI-beamer}

\title[Hanging Title]{CMI Lab LaTeX Template}
\subtitle{Subtitle}
\author{Author}
\institute[CU]{
  Department\\
  Clarkson University
}
\date{Date}
\logo{
\includegraphics[width=.4in]{img/CMI.png}
}
\subject{Human Disease Network}

\begin{document}

\maketitle

\section{Items}

\begin{frame}{Frame}
\begin{block}{Block}
\begin{itemize}
    \item item 1
    \item item 2
    \item item 3
\end{itemize}
\end{block}
\begin{itemize}
    \item item 4
    \item item 5
\end{itemize}

\end{frame}


\section{Section 2}

\begin{frame}{Image}
    \begin{figure}
        \centering
        \includegraphics[width=2in]{img/CMI.png}
        \caption{Caption}

    \end{figure}
\end{frame}

\section{Math Environment}

\begin{frame}{Math Environment}

\begin{block}{Numbered VS Non-numbered Equations}
\begin{equation}
    \theta = \int_0^{\infty}(Numbered)dx
\end{equation}
\begin{equation*}
    \theta = \int_0^{\infty}(Non-numbered)dx
\end{equation*}
\end{block}

\end{frame}

\section{Example Frame}

\begin{frame}{Infinitesimal Perturbation Analysis (IPA)}
\begin{itemize}
    \item IPA is a simulation tool for discrete event systems
    \item Main idea is that we can make small changes to certain input parameters of a system without changing the sequence in which events occur (Deterministic Similarity)\footnotemark
    \item This can be used to efficiently estimate the gradient of a surface \footnotemark
\end{itemize}
\footnotetext[1]{\tiny Wardi, Y \& Melamed, B. (1996). IPA Gradient Estimation for Loss Measures in Continuous Flow Models.}
\footnotetext[2]{\tiny Wang et al. Parallel Bayesian Global Optimization of expensive Functions. 2017, https://arxiv.org/abs/1602.05149}
\begin{figure}
    \centering
    \includegraphics[width = 3 in]{img/Perturbation.PNG}
\end{figure}
\end{frame}

\end{document}
